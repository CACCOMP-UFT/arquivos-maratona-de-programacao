% Arquivo LaTeX do caderno de problemas para competições de programação a serem realizadas na UFT 
%
% Versão 0.1: Thu May 28 BRT 2015
%
% Criado por: Herinson Rodrigues
%

\documentclass[11pt,oneside,a4paper]{book}

\usepackage[T1]{fontenc}
\usepackage[brazil]{babel}
\usepackage[utf8]{inputenc}

\usepackage{graphicx}
\usepackage{fancyhdr}

\usepackage{listings}
\usepackage{titletoc}
\usepackage[bf,small,compact]{titlesec}

\usepackage[font=small,format=plain,labelfont=bf,up,textfont=it,up]{caption}

\usepackage[a4paper,top=3cm,bottom=2.0cm,left=3.0cm,right=2.cm]{geometry}
\usepackage[pdftex,plainpages=false,pdfpagelabels,pagebackref,colorlinks=true,citecolor=black,linkcolor=black,urlcolor=black,filecolor=black,bookmarksopen=true]{hyperref}
\usepackage[numbers]{natbib}
\usepackage{enumerate}
\usepackage{lastpage}

\graphicspath{{./img/}}
\pagestyle{fancy}
\fancyhead[L]{\textit{Nome da Competição -- CC/UFT -- 2015}}

\frenchspacing                    
\raggedbottom                      
\fontsize{60}{62}\usefont{OT1}{cmr}{m}{n}{\selectfont}
\cleardoublepage
\normalsize

\begin{document}

\begin{titlepage}

\begin{center}
\includegraphics[scale=0.8]{logouft.jpg}
\hspace{3em}
\includegraphics[scale=0.11]{logoccomp.jpg}
\hspace{3em}
\includegraphics[scale=0.09]{logocaccomp.jpg}
\hspace{3em}
\includegraphics[scale=0.19]{logoacm.png}


\vspace{2em}
\large{\textbf{I Treinamento para a Maratona de Programação}} \\

\vspace{1em}
\textit{\today} \\

\vspace{3em}
\textbf{Caderno de Problemas} \\
\end{center}

\vspace{7em}
\begin{center}
\textbf{Informações Gerais} \\
\end{center}

\noindent
Este caderno contém $N$ problemas; as páginas estão numeradas de 1 a \pageref{LastPage}, não contando esta página de rosto.
Verifique se o caderno está completo. \\

\textbf{A) Sobre a entrada}
\begin{enumerate}
\setlength{\itemsep}{0pt}
\setlength{\parskip}{0pt}
\setlength{\parsep}{0pt} 
\item A entrada de seu programa deve ser lida da entrada padrão.
\item A entrada é composta de um único ou vários casos de teste, descrito em um número de linhas que depende do problema.
\item Quando uma linha da entrada contém vários valores, estes são separados por um único espaço em branco; a
entrada não contém nenhum outro espaço em branco.
\item Cada linha, incluindo a última, contém exatamente um caractere final-de-linha.
\item O final da entrada coincide com o final do arquivo.
\end{enumerate} 

\textbf{B) Sobre a saída}
\begin{enumerate}
\setlength{\itemsep}{0pt}
\setlength{\parskip}{0pt}
\setlength{\parsep}{0pt} 
\item A saída de seu programa deve ser escrita na saída padrão.
\item Quando uma linha da saída contém vários valores, estes devem ser separados por um único espaço em branco; a saída não deve conter nenhum outro espaço em branco.
\item Cada linha, incluindo a última, deve conter exatamente um caractere final-de-linha.
\end{enumerate}

\end{titlepage}

\input{problem1}
\input{problem2}
\input{problem3}
\input{problem4}
\input{problem5}
\input{problem6}
\input{problem7}
\newpage

\begin{center}
\LARGE{\textbf{Problema H} \\ \textbf{Nome completo do problema}} 

\vspace{0.5em}
\normalsize

\textit{Arquivo:}
\verb+nome_do_arquivo.[c|cpp|java]+  

\end{center}

Descrição do problema. \\

\section*{Entrada}

Descrição da entrada. \\

\section*{Saída}

Descrição da saída. \\

\begin{center}
  \begin{tabular}{ | l | l | }
    \hline
    \textbf{Exemplo de entrada} & \textbf{Exemplo de saída} \\ 
    \hline
	
	\hline
  \end{tabular}
\end{center}
\input{problem9}
\input{problem10}

\end{document}